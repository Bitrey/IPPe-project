\documentclass{article}
\usepackage[utf8]{inputenc}
\usepackage[T1]{fontenc}
\usepackage{graphicx}
\usepackage[a4paper, total={6in, 10in}]{geometry}
\setcounter{secnumdepth}{4}
\usepackage{hyperref}
\usepackage{color}
\usepackage{booktabs}
\usepackage{tabularx}
\usepackage{amsmath}
\usepackage{xcolor}
\usepackage{listings}
\usepackage{enumitem}
\usepackage{minted}
\usepackage{tikz}
\usepackage{ccicons}
\usepackage{fancyhdr}
\usepackage{pgfplots}
\usepackage[
type={CC},
modifier={by-nc-sa},
version={4.0},
]{doclicense}
\pgfplotsset{compat=newest}
\usetikzlibrary{arrows.meta, positioning}

\title{Documentation for IPPeCode Parser}

\title{%
  Documentation for IPPeCode Parser \\
  \large Task 1 - parser.py}
\author{Alessandro Amella\\xamella00}
\date{\today}

\makeindex

\pagestyle{fancy}
\fancyhf{} % Clear header and footer
\fancyhead[C]{\nouppercase{\leftmark}} % Current section at the top
\fancyfoot[C]{\thepage} % Page numbers at the bottom
\renewcommand{\headrulewidth}{0.4pt} % Header rule
\renewcommand{\footrulewidth}{0pt} % Footer rule

\begin{document}

\maketitle

\section{Introduction}
The script \texttt{parser.py} is designed to tokenize and parse IPPeCode code, converting it into an XML format. It utilizes the PLY (Python Lex-Yacc) library for lexical analysis (tokenization) and syntax analysis (parsing), and uses the Python Standard Library's XML modules for managing XML.

\section{Lexical Analysis}
Lexical analysis is performed using PLY's \texttt{lex} module. The script defines a set of tokens which includes the language operands, along with regular expressions for each.

\section{Syntax Analysis}
For syntax analysis, the script uses PLY's \texttt{yacc} module. Production rules in the script define the structure of valid IPPeCode syntax, beginning with a top-level \texttt{program} rule that consists of a list of three-address codes (TAC) elements.

An abstract syntax tree is built by grouping these into actions with specific types and values.

\section{XML Output Generation}
The script generates XML output based on the parsed data. It constructs an XML document using the ElementTree API, creating elements for the program and each TAC with their operands. XML attributes and text content are properly escaped.

\section{Debugging and Error Handling}
The script offers debug mode for detailed process logging, including tokenization and parsing steps, via command-line arguments. Additionally, it implements error handling mechanisms to report illegal characters during lexing and unexpected tokens during parsing.

\section{Extensions}
Within the project, support for string variables has been implemented to enhance the functionality of the parser: OPCODES \texttt{CONCAT}, \texttt{GETAT}, \texttt{LEN}, \texttt{STRINT}, and \texttt{INTSTR} have been added, allowing operations involving string concatenation, accessing characters by index, length determination, and conversions between strings and integers.

\section{Conclusion}
This is a robust implementation of the IPPeCode parser which, by building a context-free grammar and an abstract syntax tree, allows easy future expansion with an expanded set of instructions.

\end{document}
